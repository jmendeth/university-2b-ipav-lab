\input{preamble.tex}
\AfterPreamble{\hypersetup{
  pdftitle={Session 2, homework},
}}

\begin{document}
\startpage

\paragraph{Question 1.}

The quantizer is implemented in Matlab/Octave as:

\begin{minted}{matlab}
function y = quantize(x)
  y = round(x * 7) / 7;
  y = max(min(y, 1), -1);
end
\end{minted}

The quantized sequence is:

\begin{minted}{matlab}
>> x = [0.7, -0.1, 0.4, 0.6, -1.2];
>> xq = quantize(x)
xq =
   0.71429  -0.14286   0.42857   0.57143  -1.00000
\end{minted}

\paragraph{Question 2.}

The error sequence is:

\begin{minted}{matlab}
>> e = x - xq
e =
  -0.014286   0.042857  -0.028571   0.028571  -0.200000
\end{minted}

\paragraph{Question 3.}

The SNR in \si{\decibel} is:

\begin{minted}{matlab}
>> SNR = 10 * log10(sum(x.^2) / sum(e.^2))
SNR =  17.507
\end{minted}

\paragraph{Question 4.}

This expression doesn't apply because it does \emph{not} consider
saturation error, only granular error. $x[n]$ does have a sample outside
$[-1, 1]$ and therefore presents saturation error when quantized.

\paragraph{Question 5.}

The length of the sequence is the sum of all bins in the histogram:
$10 + 0 + 5 + 20 + 40 + 50 + 100 + 50 + 30 + 20 + 0 + 5 + 5 = 335$
samples.

By adding the samples in the corresponding bins, we can see that 270
of those samples fall in the range $[\num{-0.2}, \num{0.2}]$. Thus, their
relative frequency (probability) is $\frac{270}{370} \simeq
\SI{80.60}{\percent}$.

\paragraph{Question 6.}

\emph{Green} light is the most visible under photopic circumstances,
as its average wavelength (\SI{534}{\nano\meter}) as well as its range,
have the greatest luminous efficiency compared to the other kinds of
light mentioned.

\paragraph{Question 7.}

\begin{center}
\begin{tabular}{lccc}
\toprule
Color & R & G & B \\
\midrule
white   & X & X & X \\
red     & X &   &   \\
green   &   & X &   \\
blue    &   &   & X \\
yellow  & X & X &   \\
magenta & X &   & X \\
cyan    &   & X & X \\
\bottomrule
\end{tabular}
\end{center}

\finishpage
\end{document}
